%% LaTeX Beamer presentation template (requires beamer package)
%% see http://bitbucket.org/rivanvx/beamer/wiki/Home
%% idea contributed by H. Turgut Uyar
%% template based on a template by Till Tantau
%% this template is still evolving - it might differ in future releases!

%% Template edited by Panagiotis Adamopoulos {padamopo}@stern.nyu.edu

\documentclass{beamer}
 
\mode<presentation>
{
\usetheme{NYU}

\setbeamercovered{transparent}
}

\usepackage[english]{babel}
\usepackage[latin1]{inputenc}

% font definitions, try \usepackage{ae} instead of the following
% three lines if you don't like this look
\usepackage{mathptmx}
\usepackage[scaled=.90]{helvet}
\usepackage{courier}


\usepackage[T1]{fontenc}


\usepackage{comment}
%usepackage{appendixnumberbeamer}
%\usepackage{amsmath}
\usepackage{pgfpages}
% citations
\usepackage{natbib}
\bibpunct{(}{)}{;}{a}{,}{,}
\def\citeapos#1{\citeauthor{#1}'s (\citeyear{#1})}
\renewcommand{\bibsection}{\subsubsection*{\bibname } }

\title{Quality Assurance In Microservice Architectures}

%\subtitle{}

% - Use the \inst{?} command only if the authors have different
%   affiliation.
%\author{F.~Author\inst{1} \and S.~Another\inst{2}}
\author{Krishnan Chandran \and Irina Barykina} 

% - Use the \inst command only if there are several affiliations.
% - Keep it simple, no one is interested in your street address.
\institute[NYU]
{
Department of Informatics,\\
Intelligent Adaptive Systems, UHH\\
}

\date{2016}


% This is only inserted into the PDF information catalog. Can be left
% out.
\subject{Subject}



% If you have a file called "university-logo-filename.xxx", where xxx
% is a graphic format that can be processed by latex or pdflatex,
% resp., then you can add a logo as follows:

% \pgfdeclareimage[height=0.5cm]{university-logo}{university-logo-filename}
% \logo{\pgfuseimage{university-logo}}



% Delete this, if you do not want the table of contents to pop up at
% the beginning of each subsection:
%\AtBeginSubsection[]
%{
%\begin{frame}<beamer>
%\frametitle{Outline}
%\tableofcontents[currentsection,currentsubsection]
%\end{frame}
%}

% If you wish to uncover everything in a step-wise fashion, uncomment
% the following command:

%\beamerdefaultoverlayspecification{<+->}

\begin{document}

\begin{frame}
\titlepage
\end{frame}

%\begin{frame}
%\frametitle{Outline}
%\tableofcontents
% You might wish to add the option [pausesections]
%\end{frame}

%==============================Start
% Theory section
\section{Test pyramid and challenges}
\subsection[Short First Subsection Name]{First Subsection Name}

\begin{frame}
	\frametitle{}	
	\framesubtitle{Subtitles are optional}

	\begin{itemize}
 		 \item
		 \item
	\end{itemize}
\end{frame}
%===============================End


%==============================Start
% Continuous delivery section
\section{Continuous Delivery}
\begin{frame}
	\frametitle{Continuous Delivery} 
	\begin{block}{Martin Fowler's definition:}
		 Continuous delivery is a software development discipline where you build software in such a way that it can be released to production at any time. Principle benefits: reduced deployment risk, transparency, immediate user feedback.	
	\end{block}
	\begin{itemize}
		 \item Key prerequisites for CD: extensive automation of all possible parts of the delivery process (DeploymentPipeline) and close collaboration between everyone involved in delivery (DevOpsCulture) 
  		 \item Drawbacks of monolithic architecture from CD perspective: long build time, delayed feedback (because of large number of people checking in changes simultaneously), unclear responsibility for broken builds or bugs, time consuming test runs (automatic and manual), small change can affect the whole application. However, it is not impossible to maintain CD with monolithic architecture (example: Facebook).
                 \item Microservices are by nature loosely coupled and can be delivered relatively independently. However, they require high quality, reliable monitoring and operation infrastructure (instead of one applications we run tens of services)
	\end{itemize}
\end{frame}
%===============================End



%==============================Start
% Tools section
\section{Tools}

% Unit testing
\begin{frame}
	\frametitle{Unit Testing}
	\framesubtitle{}
	Unit testing is a testing of smallest testable parts of applications, ideally of single methods or procedures. Unit testing helps to identify bugs on the early stages of development and most precisely (up to the line of code).  
	\begin{itemize}
 		 \item xUnit family of testing frameworks derived from SmallTalk testing framework, designed by Kent Beck. Later was implemented in many other languages, including Java (JUnit), R (RUnit), JQuery (QUnit) and etc. xUnit frameworks share same component architecture.
	\end{itemize}
\end{frame}

% Contract testing
\begin{frame}
	\frametitle{Contract testing}
	\framesubtitle{}

	\begin{itemize}
 		\item Integration contract testing: performed by using of test doubles of service that is to be consumed
			\begin{itemize}
				\item Mocking 
				\item Stubbing
			\end{itemize}
		\item Consumer driven contract testing: performed by provider against consumer defined contracts
			\begin{itemize}
				\item Pact: Analogies: Pacto, Janus
				\item Data passing between services like SOAP UI
			\end{itemize}
	\end{itemize}
\end{frame}

% End-to-end testing
\begin{frame}
	\frametitle{End-to-end testing}
	\framesubtitle{Subtitles are optional}
	\begin{itemize}
 		 \item End-to-end testing tools
	\end{itemize}
\end{frame}
%===============================End

%==============================Start
% Scenarios section
\section{Scenarios}
\begin{frame}
	\frametitle{Scenarios}
	<The UML diagram of example architecture>
	\begin{itemize}
 		 \item Scenario 1: Testing microservices within application. \\
			 
		 \item Scenario 2: Testing microservices that use third-party service
		 \item Scenario 3: Testing of microservice that will be or is already exposed to public domain
	\end{itemize}
\end{frame}
%===============================End

%==============================Start
% References section
\begin{frame}
	\frametitle{References}
	\framesubtitle{}
	\begin{itemize}
 		 \item Books
		 \item Articles and blogs
	\end{itemize}
\end{frame}
%===============================End

%==============================Start
% Summary section - we do not need summary slide right not, so it commented out
%\section*{Summary}

%\begin{frame}
%\frametitle<presentation>{Summary}

%\begin{itemize}
%  \item The \alert{first main message} of your talk in one or two lines.
%\end{itemize}

% The following outlook is optional.
%\vskip0pt plus.5fill
%\begin{itemize}
% \item Outlook
% \begin{itemize}
% \item Something you haven't solved.
% \item Something else you haven't solved.
% \end{itemize}
% \end{itemize}
% \end{frame}
%===============================End

\end{document}
