%% LaTeX Beamer presentation template (requires beamer package)
%% see http://bitbucket.org/rivanvx/beamer/wiki/Home
%% idea contributed by H. Turgut Uyar
%% template based on a template by Till Tantau
%% this template is still evolving - it might differ in future releases!

%% Template edited by Panagiotis Adamopoulos {padamopo}@stern.nyu.edu

\documentclass{beamer}
 
\mode<presentation>
{
\usetheme{NYU}

\setbeamercovered{transparent}
}

\usepackage[english]{babel}
\usepackage[latin1]{inputenc}

% font definitions, try \usepackage{ae} instead of the following
% three lines if you don't like this look
\usepackage{mathptmx}
\usepackage[scaled=.90]{helvet}
\usepackage{courier}

\usepackage{float}
\usepackage{wrapfig}


\usepackage[T1]{fontenc}

\usepackage{comment}
%usepackage{appendixnumberbeamer}
%\usepackage{amsmath}
\usepackage{pgfpages}
% citations

\usepackage{natbib}
\bibliographystyle{unsrtnat}
\bibpunct{(}{)}{;}{a}{,}{,}
\def\citeapos#1{\citeauthor{#1}'s (\citeyear{#1})}
\renewcommand{\bibsection}{\subsubsection*{\bibname } }
\graphicspath{ {images/} }


\title{Quality Assurance In Microservice Architectures}

%\subtitle{}

% - Use the \inst{?} command only if the authors have different
%   affiliation.
%\author{F.~Author\inst{1} \and S.~Another\inst{2}}
\author{Krishnan Chandran \and Irina Barykina} 

% - Use the \inst command only if there are several affiliations.
% - Keep it simple, no one is interested in your street address.
\institute[NYU]
{
Department of Informatics,\\
Intelligent Adaptive Systems, UHH\\
}

\date{2016}


% This is only inserted into the PDF information catalog. Can be left
% out.
\subject{Subject}



% If you have a file called "university-logo-filename.xxx", where xxx
% is a graphic format that can be processed by latex or pdflatex,
% resp., then you can add a logo as follows:

% \pgfdeclareimage[height=0.5cm]{university-logo}{university-logo-filename}
% \logo{\pgfuseimage{university-logo}}



% Delete this, if you do not want the table of contents to pop up at
% the beginning of each subsection:
%\AtBeginSubsection[]
%{
%\begin{frame}<beamer>
%\frametitle{Outline}
%\tableofcontents[currentsection,currentsubsection]
%\end{frame}
%}

% If you wish to uncover everything in a step-wise fashion, uncomment
% the following command:

%\beamerdefaultoverlayspecification{<+->}

\begin{document}

\begin{frame}
\titlepage
\end{frame}

%\begin{frame}
%\frametitle{Outline}
%\tableofcontents
% You might wish to add the option [pausesections]
%\end{frame}

%==============================
% Theory section
%==============================
\section{Outline}
\begin{frame}
	\frametitle{Outline}	
	\framesubtitle{}
	\begin{itemize}
		\item What is Quality Assurance?
 		\item QA is easy, isn't it?
		\item QA on Development stage.
		\item QA on Deployment stage.
		\item QA after Release.
		\item Conclusion.
	\end{itemize}
\end{frame}

%===============================
% Introduction
%===============================
\section{Introduction}
\begin{frame}
	\frametitle{Introduction}	
	\framesubtitle{}

\begin{definition}
Quality Assurance refers to planned and systematic production processes that provide confidence in a product's suitability for its intended purposes.
\end{definition}
	\begin{itemize}
 		\item QA must prevent bugs and failures, not identify them.
		\item QA is wasteful on the last stages of development cycle.
	\end{itemize}
\end{frame}

%===============================
% Challenges
%===============================
\section{Challenges}
\begin{frame}
	\frametitle{Challenges}	
	\framesubtitle{}

	\begin{itemize}
 		\item 
	\end{itemize}
\end{frame}

%===============================
% CD related slides
%===============================
\section{Deployment}

\begin{frame}
	\frametitle{Deployment}
	\framesubtitle{RAD and Deployment Pipeline}

\end{frame}

\begin{frame}
	\frametitle{Deployment}
	\framesubtitle{Continuous Deployment and Delivery}

\end{frame}

\begin{frame}
	\frametitle{Deployment}
	\framesubtitle{DevOps Culture}

DevOps Culture:
	\begin{itemize}
		\item Aim: break silos between development and later stages 
		\item Requirements: shared responsibility and autonomy of teams
	\end{itemize}
	\begin{figure}
		\begin{center}
% image is from http://blogs.atlassian.com/2016/03/how-to-choose-devops-tools/
% reference it!
 			\includegraphics[scale=0.12]{devopsloop}
		\end{center}
	\end{figure}

\end{frame}


%===============================Start
% Smart releases slides
%===============================End
\section{After Deployment}

\begin{frame}
	\frametitle{After Deployment}
	\framesubtitle{Smart releasing strategies}
\begin{columns}
 \begin{column}{.49\textwidth}
	\begin{itemize}
		\item Smoke Test Suites
		\item Blue/Green Deployment
		\item Canary releasing
	\end{itemize}
\end{column}
\begin{column}{.49\textwidth}
	\begin{figure}
		\begin{center}
 			\only<1>{\includegraphics[ scale=0.16]{blue_green}\\}
 			\only<2>{\includegraphics[ scale=0.16]{canary}}\par
		\end{center}
	\end{figure}
\end{column}
\end{columns}
\end{frame}

\begin{frame}
	\frametitle{After Deployment}
	\framesubtitle{Monitoring}
\end{frame}

%==============================
% Tools section
%==============================
\section{Tools}

\begin{frame}
	\frametitle{Tools}
	\framesubtitle{}
	\begin{itemize}
		% Unit testing: Unit testing is a testing of smallest testable parts of applications, ideally of single methods or procedures. Unit testing helps to identify bugs on
		% the early stages and most precisely (up to the line of code).  Unit testing is also the powerful designing tool (especially if it is used in the context of TDD),
		% because it highlights when module should be broken into smaller more coherent pieces.
		\item xUnit framework
		\item stubbing and mocking (on the example of Mockito)
		\item smart stubbing with Mountebank
		\item testing of data passing between services (on the example of SOAP UI)
		\item consumer driven testing (on the example of Pact)
		\item End-to-End Testing (BDD Tools, JBehave, Cucumber)
	\end{itemize}
\end{frame}

%==============================Start
% References section
\section{References}
\begin{frame}
	\frametitle{References}
	\framesubtitle{}

	% make a list even without citation
        \nocite{newman,cohn,infosys,clemson,fowler_cont_del,naik}
	\bibliography{references}
\end{frame}

%===============================End


\end{document}
